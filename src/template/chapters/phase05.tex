\section{Phase 5: Intrinsic Value}

\subsection{Rule 1: Low Price-to-Earnings (P/E) ratio}

The P/E ratio should be $P/E = \frac{\text{Market Price per Share}}{\text{Earnings Per Share (EPS)}} < 15$.
This guarantees a return of at least $\frac{1}{15} = 6.67\%$.\\

\begin{tabularx}{\textwidth}{|X|X|}
 \hline
 \multicolumn{2}{|c|}{Price-to-Earnings Ratio (P/E)}                   \\
 \hline
 Market Price Per Share   & \fundData[Highlights][50DayMA] \\
 Earnings                 & \fundData[Financials][IncomeStatement][NetIncome][0]               \\
 Shares                   & \fundData[Financials][BalanceSheet][CommonSharesOutstanding][0] \\
 EPS                      & \fundData[Financials][IncomeStatement][EarningsPerShare][0]        \\
 \rowcolor{lightgray} P/E & \calcData[Ratios][PriceToEarningsRatio]                   \\
 \hline
\end{tabularx}

\subsection{Rule 2: Low Price-to-Book (P/B) ratio}

The P/B ratio should be $P/B = \frac{\text{Market Price per Share}}{\text{Book Value per Share (BVPS)}} < 1.5$.
The BVPS is defined as $BVPS = \frac{Equity}{Shares}$.\\

\begin{tabularx}{\textwidth}{|X|X|}
 \hline
 \multicolumn{2}{|c|}{Price-to-Book Ratio (P/B)}                       \\
 \hline
 Market Price Per Share   & \fundData[Highlights][50DayMA] \\
 Equity                   & \fundData[Financials][BalanceSheet][TotalEquity][0]             \\
 Shares                   & \fundData[Financials][BalanceSheet][CommonSharesOutstanding][0] \\
 BVPS                     & \fundData[Financials][BalanceSheet][BookValuePerShare][0]         \\
 \rowcolor{lightgray} P/B & \calcData[Ratios][PriceToBookRatio]                       \\
 \hline
\end{tabularx}

\subsection{Rule 3: Sum-of-the-Parts (SOTP) Valuation Method}

The sum-of-the-parts valuation (SOTP) is a process of valuing a company by
determining what its aggregate divisions would be worth if they were spun off or
acquired by another company. It is also known as breakup value analysis.\\
The valuation provides a range of values for a company's equity by aggregating
the standalone value of each of its business units and arriving at a single
total enterprise value (TEV). The equity value is then derived by adjusting the
company's net debt and other non-operating assets and expenses.\\
The formula for SOTP is as follows: $SOTP = N_1 + N_2 + ... + ND - NL + NA$ with:
\begin{itemize}
    \item $N_1$: value of first segment
    \item $N_2$: value of second segment
    \item $ND$: net debt
    \item $NL$: nonoperating liabilities
    \item $NA$: nonoperating assets
\end{itemize}

More information can be found at:
\begin{itemize}
    \item \url{https://www.investopedia.com/terms/s/sumofpartsvaluation.asp}
    \item \url{https://www.investopedia.com/terms/b/breakup-value.asp}
\end{itemize}

\subsection{Rule 4: DCF Valuation Method}

Discounted Cash Flow (DCF) Analysis:
\begin{enumerate}
	\item Determine the current (base year) free cash flow: $BYFCF = \text{Operating Cash Flow} + \text{CAPEX}$:\\\\
\begin{tabularx}{\textwidth}{|X|X|}
 \hline
 \multicolumn{2}{|c|}{Base Year Free Cash Flow (BYFCF)} \\
 \hline
 Operating Cash Flow        & \fundData[Financials][CashFlowStatement][OperatingCashFlow][0]   \\
 CAPEX                      & \fundData[Financials][CashFlowStatement][CapitalExpenditures][0] \\
 \rowcolor{lightgray} BYFCF & \fundData[Financials][CashFlowStatement][FreeCashFlow][0]        \\
 \hline
\end{tabularx}
	\item Estimate the free cash flow for the next 10 years: $FCF_n = BYFCF \cdot (1+GR)^n$:
	\begin{itemize}
        \item BYFCF: \fundData[Financials][CashFlowStatement][FreeCashFlow][0]
        \item (Historical) FCF Growth Rate (GR): \calcData[DCF][FCFHistoricalGrowthRate]
    \end{itemize}
\begin{tabularx}{\textwidth}{|X|X|X|}
 \hline
 \multicolumn{3}{|c|}{$\text{FCF}_n$} \\
 \hline
 n & 1 & 2 \\
 \hline
    \rowcolor{lightgray} $\text{FCF}_n$ & \calcData[DCF][FCFFutureEstimate][1] & \calcData[DCF][FCFFutureEstimate][2] \\
 \hline
\end{tabularx}\\

\begin{tabularx}{\textwidth}{|X|X|X|}
 \hline
 n & 3 & 4 \\
 \hline
    \rowcolor{lightgray} $\text{FCF}_n$ & \calcData[DCF][FCFFutureEstimate][3] & \calcData[DCF][FCFFutureEstimate][4] \\
 \hline
\end{tabularx}\\

\begin{tabularx}{\textwidth}{|X|X|X|}
 \hline
 n & 5 & 6 \\
 \hline
    \rowcolor{lightgray} $\text{FCF}_n$ & \calcData[DCF][FCFFutureEstimate][5] & \calcData[DCF][FCFFutureEstimate][6] \\
 \hline
\end{tabularx}\\

\begin{tabularx}{\textwidth}{|X|X|X|}
 \hline
 n & 7 & 8 \\
 \hline
    \rowcolor{lightgray} $\text{FCF}_n$ & \calcData[DCF][FCFFutureEstimate][7] & \calcData[DCF][FCFFutureEstimate][8] \\
 \hline
\end{tabularx}\\

\begin{tabularx}{\textwidth}{|X|X|X|}
 \hline
 n & 9 & 10 \\
 \hline
    \rowcolor{lightgray} $\text{FCF}_n$ & \calcData[DCF][FCFFutureEstimate][9] & \calcData[DCF][FCFFutureEstimate][10] \\
 \hline
\end{tabularx}\\
	\item Estimate the discount factor for the next 10 years: $DF_n = (1+DR)^n$:
	\begin{itemize}
        \item Discount Rate (DR): \calcData[DCF][DiscountRate]
    \end{itemize}
\begin{tabularx}{\textwidth}{|X|X|X|}
 \hline
 \multicolumn{3}{|c|}{$\text{DF}_n$} \\
 \hline
 n & 1 & 2 \\
 \hline
    \rowcolor{lightgray} $\text{DF}_n$ & \calcData[DCF][DiscountFactorEstimate][1] & \calcData[DCF][DiscountFactorEstimate][2] \\
 \hline
\end{tabularx}\\

\begin{tabularx}{\textwidth}{|X|X|X|}
 \hline
 n & 3 & 4 \\
 \hline
    \rowcolor{lightgray} $\text{DF}_n$ & \calcData[DCF][DiscountFactorEstimate][3] & \calcData[DCF][DiscountFactorEstimate][4] \\
 \hline
\end{tabularx}\\

\begin{tabularx}{\textwidth}{|X|X|X|}
 \hline
 n & 5 & 6 \\
 \hline
    \rowcolor{lightgray} $\text{DF}_n$ & \calcData[DCF][DiscountFactorEstimate][5] & \calcData[DCF][DiscountFactorEstimate][6] \\
 \hline
\end{tabularx}\\

\begin{tabularx}{\textwidth}{|X|X|X|}
 \hline
 n & 7 & 8 \\
 \hline
    \rowcolor{lightgray} $\text{DF}_n$ & \calcData[DCF][DiscountFactorEstimate][7] & \calcData[DCF][DiscountFactorEstimate][8] \\
 \hline
\end{tabularx}\\

\begin{tabularx}{\textwidth}{|X|X|X|}
 \hline
 n & 9 & 10 \\
 \hline
    \rowcolor{lightgray} $\text{DF}_n$ & \calcData[DCF][DiscountFactorEstimate][9] & \calcData[DCF][DiscountFactorEstimate][10] \\
 \hline
\end{tabularx}\\
	\item Calculate the discounted value of FCF for the next 10 years: $DFCF_n = \frac{FCF_n}{DF_n}$:\\
\begin{tabularx}{\textwidth}{|X|X|X|}
 \hline
 \multicolumn{3}{|c|}{$\text{DFCF}_n$} \\
 \hline
 n & 1 & 2 \\
 \hline
 $\text{FCF}_n$                       & \calcData[DCF][FCFFutureEstimate][1]      & \calcData[DCF][FCFFutureEstimate][2]      \\
 $\text{DF}_n$                        & \calcData[DCF][DiscountFactorEstimate][1] & \calcData[DCF][DiscountFactorEstimate][2] \\
 \rowcolor{lightgray} $\text{DFCF}_n$ & \calcData[DCF][DiscountedCashFlow][1]     & \calcData[DCF][DiscountedCashFlow][2]     \\
 \hline
\end{tabularx}\\

\begin{tabularx}{\textwidth}{|X|X|X|}
 \hline
 n & 3 & 4 \\
 \hline
 $\text{FCF}_n$                       & \calcData[DCF][FCFFutureEstimate][3]      & \calcData[DCF][FCFFutureEstimate][4]      \\
 $\text{DF}_n$                        & \calcData[DCF][DiscountFactorEstimate][3] & \calcData[DCF][DiscountFactorEstimate][4] \\
 \rowcolor{lightgray} $\text{DFCF}_n$ & \calcData[DCF][DiscountedCashFlow][3]     & \calcData[DCF][DiscountedCashFlow][4]     \\
 \hline
\end{tabularx}\\

\begin{tabularx}{\textwidth}{|X|X|X|}
 \hline
 n & 5 & 6 \\
 \hline
 $\text{FCF}_n$                       & \calcData[DCF][FCFFutureEstimate][5]      & \calcData[DCF][FCFFutureEstimate][6]      \\
 $\text{DF}_n$                        & \calcData[DCF][DiscountFactorEstimate][5] & \calcData[DCF][DiscountFactorEstimate][6] \\
 \rowcolor{lightgray} $\text{DFCF}_n$ & \calcData[DCF][DiscountedCashFlow][5]     & \calcData[DCF][DiscountedCashFlow][6]     \\
 \hline
\end{tabularx}\\

\begin{tabularx}{\textwidth}{|X|X|X|}
 \hline
 n & 7 & 8 \\
 \hline
 $\text{FCF}_n$                       & \calcData[DCF][FCFFutureEstimate][7]      & \calcData[DCF][FCFFutureEstimate][8]      \\
 $\text{DF}_n$                        & \calcData[DCF][DiscountFactorEstimate][7] & \calcData[DCF][DiscountFactorEstimate][8] \\
 \rowcolor{lightgray} $\text{DFCF}_n$ & \calcData[DCF][DiscountedCashFlow][7]     & \calcData[DCF][DiscountedCashFlow][8]     \\
 \hline
\end{tabularx}\\

\begin{tabularx}{\textwidth}{|X|X|X|}
 \hline
 n & 9 & 10 \\
 \hline
 $\text{FCF}_n$                       & \calcData[DCF][FCFFutureEstimate][9]      & \calcData[DCF][FCFFutureEstimate][10]      \\
 $\text{DF}_n$                        & \calcData[DCF][DiscountFactorEstimate][9] & \calcData[DCF][DiscountFactorEstimate][10] \\
 \rowcolor{lightgray} $\text{DFCF}_n$ & \calcData[DCF][DiscountedCashFlow][9]     & \calcData[DCF][DiscountedCashFlow][10]     \\
 \hline
\end{tabularx}\\

	\item Calculate the discounted perpetuity free cash flow (beyond 10 years):
	$DPCF = \frac{BYFCF \cdot (1+GR)^{11} \cdot (1+LGR)}{DR-LGR} \cdot \frac{1}{(1+DR)^{11}}$.
	The long-term growth rate (LGR) should be at 3\%.\\\\
\begin{tabularx}{\textwidth}{|X|X|}
 \hline
 \multicolumn{2}{|c|}{Discounted Perpetuity FCF (DPCF)} \\
 \hline
 BYFCF                             & \fundData[Financials][CashFlowStatement][FreeCashFlow][0] \\
 (Historical) FCF Growth Rate (GR) & \calcData[DCF][FCFHistoricalGrowthRate] \\
 Long-Term Growth Rate (LGR)       & \calcData[DCF][LongTermGrowthRate] \\
 Discount Rate (DR)                & \calcData[DCF][DiscountRate] \\
 \rowcolor{lightgray} DPCF         & \calcData[DCF][DiscountedPerpetuityCashFlow] \\
 \hline
\end{tabularx}\\
	\item Calculate the intrinsic value: $\text{Intrinsic Value} = (\sum_{i=1}^n DFCF_n) + DPCF$:\\\\
\begin{tabularx}{\textwidth}{|X|X|}
 \hline
 \multicolumn{2}{|c|}{Intrinsic Value} \\
 \hline
 $\sum_{i=1}^n DFCF_n$                & \calcData[DCF][SumDiscountedCashFlow] \\
 DPCF                                 & \calcData[DCF][DiscountedPerpetuityCashFlow] \\
 \rowcolor{lightgray} Intrinsic Value & \calcData[DCF][IntrinsicValue] \\
 \hline
\end{tabularx}\\
	\item Calculate the $\text{Intrinsic Value Per Share} = \frac{\text{Intrinsic Value}}{\text{Common Shares Outstanding}}$:\\\\
\begin{tabularx}{\textwidth}{|X|X|}
 \hline
 \multicolumn{2}{|c|}{Intrinsic Value Per Share} \\
 \hline
 Intrinsic Value                                & \calcData[DCF][IntrinsicValue] \\
 Common Shares Outstanding                      & \fundData[Financials][BalanceSheet][CommonSharesOutstanding][0] \\
 \rowcolor{lightgray} Intrinsic Value Per Share & \calcData[DCF][IntrinsicValuePerShare] \\
 \hline
\end{tabularx}
\end{enumerate}

\subsection{Rule 5: Buy stock at wide Margin of Safety}

The Margin of Safety should be 50\% lower than the intrinsic value. If the market
price goes below half of the intrinsic value per share, this is a buy signal.\\

\begin{tabularx}{\textwidth}{|X|X|}
 \hline
 \multicolumn{2}{|c|}{Margin of Safety (MOS)} \\
 \hline
 Intrinsic Value Per Share & \calcData[DCF][IntrinsicValuePerShare] \\
 \rowcolor{lightgray} MOS  & \calcData[DCF][MarginOfSafety] \\
 \hline
\end{tabularx}
