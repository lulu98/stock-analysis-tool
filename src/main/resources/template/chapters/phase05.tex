\section{Phase 5: Intrinsic Value}

\subsection{Rule 1: Low Price-to-Earnings (P/E) ratio}

The P/E ratio should be $P/E = \frac{\text{Market Price per Share}}{\text{Earnings Per Share (EPS)}} < 15$.
This guarantees a return of at least $\frac{1}{15} = 6.67\%$.\\

\begin{tabularx}{\textwidth}{|X|X|}
 \hline
 \multicolumn{2}{|c|}{Price-to-Earnings Ratio (P/E)}                   \\
 \hline
 Market Price Per Share   & \fundData[Highlights][50DayMA] \\
 Earnings                 & \fundData[Financials][IncomeStatement][NetIncome][0]               \\
 Shares                   & \fundData[Financials][BalanceSheet][CommonSharesOutstanding][0] \\
 EPS                      & \fundData[Financials][IncomeStatement][EarningsPerShare][0]        \\
 \rowcolor{lightgray} P/E & \calcData[Ratios][PriceToEarningsRatio]                   \\
 \hline
\end{tabularx}

\subsection{Rule 2: Low Price-to-Book (P/B) ratio}

The P/B ratio should be $P/B = \frac{\text{Market Price per Share}}{\text{Book Value per Share (BVPS)}} < 1.5$.
The BVPS is defined as $BVPS = \frac{Equity}{Shares}$.\\

\begin{tabularx}{\textwidth}{|X|X|}
 \hline
 \multicolumn{2}{|c|}{Price-to-Book Ratio (P/B)}                       \\
 \hline
 Market Price Per Share   & \fundData[Highlights][50DayMA] \\
 Equity                   & \fundData[Financials][BalanceSheet][TotalEquity][0]             \\
 Shares                   & \fundData[Financials][BalanceSheet][CommonSharesOutstanding][0] \\
 BVPS                     & \fundData[Financials][BalanceSheet][BookValuePerShare][0]         \\
 \rowcolor{lightgray} P/B & \calcData[Ratios][PriceToBookRatio]                       \\
 \hline
\end{tabularx}

\subsection{Rule 3: Sum-of-the-Parts (SOTP) Valuation Method}

The sum-of-the-parts valuation (SOTP) is a process of valuing a company by
determining what its aggregate divisions would be worth if they were spun off or
acquired by another company. It is also known as breakup value analysis.\\
The valuation provides a range of values for a company's equity by aggregating
the standalone value of each of its business units and arriving at a single
total enterprise value (TEV). The equity value is then derived by adjusting the
company's net debt and other non-operating assets and expenses.\\
The formula for SOTP is as follows: $SOTP = N_1 + N_2 + ... + ND - NL + NA$ with:
\begin{itemize}
    \item $N_1$: value of first segment
    \item $N_2$: value of second segment
    \item $ND$: net debt
    \item $NL$: nonoperating liabilities
    \item $NA$: nonoperating assets
\end{itemize}

More information can be found at:
\begin{itemize}
    \item \url{https://www.investopedia.com/terms/s/sumofpartsvaluation.asp}
    \item \url{https://www.investopedia.com/terms/b/breakup-value.asp}
\end{itemize}

\subsection{Rule 4: DCF Valuation Method}

Discounted Cash Flow (DCF) Analysis Description:
\begin{enumerate}
	\item Determine the current (base year) free cash flow: $BYFCF = \text{Operating Cash Flow} + \text{CAPEX}$.
	\item Estimate the free cash flow for the next 10 years: $FCF_n = BYFCF \cdot (1+GR)^n$.
	\item Estimate the discount factor for the next 10 years: $DF_n = (1+DR)^n$.
	\item Calculate the discounted value of FCF for the next 10 years: $DFCF_n = \frac{FCF_n}{DF_n}$.
	\item Calculate the discounted perpetuity free cash flow (beyond 10 years):
	$DPCF = \frac{BYFCF \cdot (1+GR)^{11} \cdot (1+LGR)}{DR-LGR} \cdot \frac{1}{(1+DR)^{11}}$.
	The long-term growth rate (LGR) should be at 3\%.
	\item Calculate the intrinsic value: $\text{Intrinsic Value} = (\sum_{i=1}^n DFCF_n) + DPCF$.
	\item Calculate the $\text{Intrinsic Value Per Share} = \frac{\text{Intrinsic Value}}{\text{Common Shares Outstanding}}$.
    \item Buy at a Wide Margin of Safety (MOS): $\text{MOS} = 0.5 * \text{Intrinsic Value Per Share}$.
        The Margin of Safety should be 50\% lower than the intrinsic value. If 
        the market price goes below half of the intrinsic value per share, this 
        is a buy signal.\\
\end{enumerate}

Discounted Cash Flow (DCF) Analysis Calculation:
\begin{itemize}
    \item Base Year Free Cash Flow (BYFCF): \calcData[DCF][BaseYearFCF]
    \item Long-Term Growth Rate (LGR): \calcData[DCF][LongTermGrowthRate]
    \item Historical Growth Rate (GR): \calcData[DCF][FCFHistoricalGrowthRate]
    \item Discount Rate (DR): The higher the DR, the more risk there is to the 
        company.
    \item 10\% discount rate: if low risk of investment \\\\
\begin{tabularx}{\textwidth}{|X|X|X|}
 \hline
 \multicolumn{3}{|c|}{DCF Low Risk Estimates} \\
 \hline
    Growth Rates & Intrinsic Value Per Share & Margin Of Safety (MOS) \\
 \hline
    -15\% & \calcData[DCF][LowRisk][IntrinsicValuePerShare][-15Per] & \calcData[DCF][LowRisk][MarginOfSafety][-15Per] \\
    -10\% & \calcData[DCF][LowRisk][IntrinsicValuePerShare][-10Per] & \calcData[DCF][LowRisk][MarginOfSafety][-10Per] \\
    -5\% & \calcData[DCF][LowRisk][IntrinsicValuePerShare][-5Per] & \calcData[DCF][LowRisk][MarginOfSafety][-5Per] \\
    0\% & \calcData[DCF][LowRisk][IntrinsicValuePerShare][0Per] & \calcData[DCF][LowRisk][MarginOfSafety][0Per] \\
    5\% & \calcData[DCF][LowRisk][IntrinsicValuePerShare][5Per] & \calcData[DCF][LowRisk][MarginOfSafety][5Per] \\
    10\% & \calcData[DCF][LowRisk][IntrinsicValuePerShare][10Per] & \calcData[DCF][LowRisk][MarginOfSafety][10Per] \\
    15\% & \calcData[DCF][LowRisk][IntrinsicValuePerShare][15Per] & \calcData[DCF][LowRisk][MarginOfSafety][15Per] \\
    \rowcolor{lightgray} Historical Growth Rate & \calcData[DCF][LowRisk][IntrinsicValuePerShare][Hist] & \calcData[DCF][LowRisk][MarginOfSafety][Hist] \\
 \hline
\end{tabularx}
    \item 15\% discount rate: if medium risk of investment \\\\
\begin{tabularx}{\textwidth}{|X|X|X|}
 \hline
 \multicolumn{3}{|c|}{DCF Medium Risk Estimates} \\
 \hline
    Growth Rates & Intrinsic Value Per Share & Margin Of Safety (MOS) \\
 \hline
    -15\% & \calcData[DCF][MediumRisk][IntrinsicValuePerShare][-15Per] & \calcData[DCF][MediumRisk][MarginOfSafety][-15Per] \\
    -10\% & \calcData[DCF][MediumRisk][IntrinsicValuePerShare][-10Per] & \calcData[DCF][MediumRisk][MarginOfSafety][-10Per] \\
    -5\% & \calcData[DCF][MediumRisk][IntrinsicValuePerShare][-5Per] & \calcData[DCF][MediumRisk][MarginOfSafety][-5Per] \\
    0\% & \calcData[DCF][MediumRisk][IntrinsicValuePerShare][0Per] & \calcData[DCF][MediumRisk][MarginOfSafety][0Per] \\
    5\% & \calcData[DCF][MediumRisk][IntrinsicValuePerShare][5Per] & \calcData[DCF][MediumRisk][MarginOfSafety][5Per] \\
    10\% & \calcData[DCF][MediumRisk][IntrinsicValuePerShare][10Per] & \calcData[DCF][MediumRisk][MarginOfSafety][10Per] \\
    15\% & \calcData[DCF][MediumRisk][IntrinsicValuePerShare][15Per] & \calcData[DCF][MediumRisk][MarginOfSafety][15Per] \\
    \rowcolor{lightgray} Historical Growth Rate & \calcData[DCF][MediumRisk][IntrinsicValuePerShare][Hist] & \calcData[DCF][MediumRisk][MarginOfSafety][Hist] \\
 \hline
\end{tabularx}
    \item 20\% discount rate: if high risk of investment \\\\
\begin{tabularx}{\textwidth}{|X|X|X|}
 \hline
 \multicolumn{3}{|c|}{DCF High Risk Estimates} \\
 \hline
    Growth Rates & Intrinsic Value Per Share & Margin Of Safety (MOS) \\
 \hline
    -15\% & \calcData[DCF][HighRisk][IntrinsicValuePerShare][-15Per] & \calcData[DCF][HighRisk][MarginOfSafety][-15Per] \\
    -10\% & \calcData[DCF][HighRisk][IntrinsicValuePerShare][-10Per] & \calcData[DCF][HighRisk][MarginOfSafety][-10Per] \\
    -5\% & \calcData[DCF][HighRisk][IntrinsicValuePerShare][-5Per] & \calcData[DCF][HighRisk][MarginOfSafety][-5Per] \\
    0\% & \calcData[DCF][HighRisk][IntrinsicValuePerShare][0Per] & \calcData[DCF][HighRisk][MarginOfSafety][0Per] \\
    5\% & \calcData[DCF][HighRisk][IntrinsicValuePerShare][5Per] & \calcData[DCF][HighRisk][MarginOfSafety][5Per] \\
    10\% & \calcData[DCF][HighRisk][IntrinsicValuePerShare][10Per] & \calcData[DCF][HighRisk][MarginOfSafety][10Per] \\
    15\% & \calcData[DCF][HighRisk][IntrinsicValuePerShare][15Per] & \calcData[DCF][HighRisk][MarginOfSafety][15Per] \\
    \rowcolor{lightgray} Historical Growth Rate & \calcData[DCF][HighRisk][IntrinsicValuePerShare][Hist] & \calcData[DCF][HighRisk][MarginOfSafety][Hist] \\
 \hline
\end{tabularx}
\end{itemize}
