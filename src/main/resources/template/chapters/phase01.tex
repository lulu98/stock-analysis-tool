\section{Phase 1: First Impression}

\subsection{The Company}

\fundData[General][Description]

\subsection{General Information}

\subsubsection{General Company Information}

\begin{tabularx}{\textwidth}{|X|X|}
 \hline
 \multicolumn{2}{|c|}{General Company Information} \\
 \hline
    Company Name     & \fundData[General][Name]          \\
    Ticker Symbol    & \fundData[General][Code]          \\
    ISIN             & \fundData[General][ISIN]          \\
    Address          & \fundData[General][Address]       \\
    Sector           & \fundData[General][Sector]        \\
    Industry         & \fundData[General][Industry]      \\
    Employees        & \fundData[General][Employees]     \\
    Fiscal Year Ends & \fundData[General][FiscalYearEnd] \\
    IPO              & \fundData[General][IPODate]       \\
 \hline
\end{tabularx}

\subsubsection{General Stock Information}

\begin{tabularx}{\textwidth}{|X|X|}
 \hline
 \multicolumn{2}{|c|}{General Stock Information} \\
 \hline
    Currency               & \fundData[General][Currency]                \\
    Market Capitalization  & \fundData[Highlights][MarketCapitalization] \\
    Dividend Share         & \fundData[Highlights][DividendShare]        \\
    Dividend Yield (in \%) & \fundData[Highlights][DividendYield]        \\
 \hline
\end{tabularx}

\subsection{Stock Categories}

\subsubsection{Definition}

Peter Lynch characterizes stocks into 6 different categories:
\begin{enumerate}
    \item Slow Growers/Sluggards: Slow growers are those stocks that have a slow 
        growth rate i.e. a low upward slope of earnings and revenue growth. The 
        growth is usually between 2-5\% CAGR and lower than the industry average. 
        These sluggards may once were fast growers, but couldn’t maintain their 
        growth rate as they grew bigger and can be characterized by the size and 
        generosity of their dividend.\\\\
        the only reason to buy these stocks are dividends. These stocks generally 
        give a decent dividend (about 2-4\%) and are a good asset during the 
        recession as it’s very unlikely for their stock to crash too hard.
    \item Stalwarts: These stocks have an average growth rate as that of industry 
        and are usually mid to large companies. They have an earnings growth 
        rate better than the Sluggards and can be typically between the 8-12 percent 
        CAGR range.\\\\
        investors can get an adequate return from these stocks if they hold 
        these stocks for a long time. They generally end up being two-baggers to 
        four-baggers i.e. they can give you 2-4 times returns in long run. Peter 
        Lynch advised that it is good to have a few stalwarts in your portfolio.
    \item Fast Growers: The fast growers are everyone’s first choice. These stocks 
        are generally aggressive companies and they grow at an impressive rate of 
        15-25\% per year. They are fast-growth stocks and grow at a comparatively 
        faster rate compared to the industry average and competitors. However, 
        Peter Lynch advises that one should be open-eyed when they own a fast 
        grower. There is a great likelihood for the fast growers to get hammered 
        if they run out of steam or if their growth is not sustainable.
    \item Cyclicals: The Cyclical are those stocks that grow at a very fast pace 
        during their favorable economic cycle. They can be distinguished from the 
        fast growers as the cyclical keeps on expanding and contracting and again 
        repeating the same cycle. On the other hand, the fast growers keep on 
        growing consistently. The cyclical companies tend to flourish when coming 
        out of a recession into a vigorous economy.\\\\
        Generally, Automobiles, Metals, Tourism, etc are examples of the cyclical 
        industry. The stock charts of these companies tend to be cyclical and go 
        up \& down over different phases of time.\\\\
        Peter Lynch advises the investors to own the cyclical only on the right 
        part of the cycle i.e. when they are expanding. If bought at the wrong 
        phase, it may even take them years before they perform. Timing is 
        everything while investing in cyclical stocks and investors need to be 
        able to detect the early signs that the industry is picking up or falling 
        down.
    \item Turnarounds: The turnarounds are characterized as potential fatalities 
        that have been badly hammered by the market for one or more of a variety 
        of reasons but can make up the lost ground under the correct circumstances.\\\\
        Holding turnarounds can be very profitable if the management is able to 
        turn the company as these stocks can be bought at a very low valuation 
        by the investors. However, if the management fails to bring back the 
        company on track, it can be very troublesome for the investors.
    \item Asset Plays: The asset plays are those stocks whose assets are 
        overlooked by the market and are undervalued. These assets may be 
        properties, equipment, or other real assets that the company is holding 
        but which is not valued by the investors when there has been a general 
        market downturn. The real value may be worth more than the market 
        capitalization of the company.\\\\
        Many of the Public sector units are key asset plays because of the real 
        estate property they are holding.\\\\
        Peter Lynch understands the worth of the asset plays. He suggests owning 
        a few of these stocks in your portfolio as they are most likely to add a 
        lot of value to your portfolio. However, the biggest significant factor 
        while picking these stocks is to carefully estimate the right worth of 
        the assets. If you are able to do it, you can pick valuable gems.
\end{enumerate}

\subsubsection{Characterization}

\subsection{Shareholder Structure}

\begin{tabularx}{\textwidth}{|X|X|}
    \hline
    \multicolumn{2}{|c|}{General Shareholder Information} \\
    \hline
    Shares Outstanding            & \fundData[SharesStats][SharesOutstanding]   \\
    Shares Floating               & \fundData[SharesStats][SharesFloat]         \\
    Float-to-Outstanding Ratio    & \calcData[Ratios][FloatToOutstandingRatio]  \\
    Owned by Institutions (in \%) & \fundData[SharesStats][PercentInstitutions] \\
    Owned by Insiders (in \%)     & \fundData[SharesStats][PercentInsiders]     \\
    Owned by Public (in \%)       & \fundData[SharesStats][PercentPublic]       \\
    \hline
\end{tabularx}

\subsubsection{Institutions}

\begin{tabularx}{\textwidth}{|X|X|X|X|}
    \hline
    \multicolumn{4}{|c|}{Institutions} \\
    \hline
    Name & Total Shares (\%) & Change (\%) & Total Assets (\%) \\
    \hline
    \fundData[Holders][Institutions][0][name] & \fundData[Holders][Institutions][0][totalShares] & \fundData[Holders][Institutions][0][change_p] & \fundData[Holders][Institutions][0][totalAssets] \\
    \fundData[Holders][Institutions][1][name] & \fundData[Holders][Institutions][1][totalShares] & \fundData[Holders][Institutions][1][change_p] & \fundData[Holders][Institutions][1][totalAssets] \\
    \fundData[Holders][Institutions][2][name] & \fundData[Holders][Institutions][2][totalShares] & \fundData[Holders][Institutions][2][change_p] & \fundData[Holders][Institutions][2][totalAssets] \\
    \fundData[Holders][Institutions][3][name] & \fundData[Holders][Institutions][3][totalShares] & \fundData[Holders][Institutions][3][change_p] & \fundData[Holders][Institutions][3][totalAssets] \\
    \fundData[Holders][Institutions][4][name] & \fundData[Holders][Institutions][4][totalShares] & \fundData[Holders][Institutions][4][change_p] & \fundData[Holders][Institutions][4][totalAssets] \\
    \fundData[Holders][Institutions][5][name] & \fundData[Holders][Institutions][5][totalShares] & \fundData[Holders][Institutions][5][change_p] & \fundData[Holders][Institutions][5][totalAssets] \\
    \fundData[Holders][Institutions][6][name] & \fundData[Holders][Institutions][6][totalShares] & \fundData[Holders][Institutions][6][change_p] & \fundData[Holders][Institutions][6][totalAssets] \\
    \fundData[Holders][Institutions][7][name] & \fundData[Holders][Institutions][7][totalShares] & \fundData[Holders][Institutions][7][change_p] & \fundData[Holders][Institutions][7][totalAssets] \\
    \fundData[Holders][Institutions][8][name] & \fundData[Holders][Institutions][8][totalShares] & \fundData[Holders][Institutions][8][change_p] & \fundData[Holders][Institutions][8][totalAssets] \\
    \fundData[Holders][Institutions][9][name] & \fundData[Holders][Institutions][9][totalShares] & \fundData[Holders][Institutions][9][change_p] & \fundData[Holders][Institutions][9][totalAssets] \\
    \hline
\end{tabularx}

\subsubsection{Funds}

\begin{tabularx}{\textwidth}{|X|X|X|X|}
    \hline
    \multicolumn{4}{|c|}{Funds} \\
    \hline
    Name & Total Shares (\%) & Change (\%) & Total Assets (\%) \\
    \hline
    \fundData[Holders][Funds][0][name] & \fundData[Holders][Funds][0][totalShares] & \fundData[Holders][Funds][0][change_p] & \fundData[Holders][Funds][0][totalAssets] \\
    \fundData[Holders][Funds][1][name] & \fundData[Holders][Funds][1][totalShares] & \fundData[Holders][Funds][1][change_p] & \fundData[Holders][Funds][1][totalAssets] \\
    \fundData[Holders][Funds][2][name] & \fundData[Holders][Funds][2][totalShares] & \fundData[Holders][Funds][2][change_p] & \fundData[Holders][Funds][2][totalAssets] \\
    \fundData[Holders][Funds][3][name] & \fundData[Holders][Funds][3][totalShares] & \fundData[Holders][Funds][3][change_p] & \fundData[Holders][Funds][3][totalAssets] \\
    \fundData[Holders][Funds][4][name] & \fundData[Holders][Funds][4][totalShares] & \fundData[Holders][Funds][4][change_p] & \fundData[Holders][Funds][4][totalAssets] \\
    \fundData[Holders][Funds][5][name] & \fundData[Holders][Funds][5][totalShares] & \fundData[Holders][Funds][5][change_p] & \fundData[Holders][Funds][5][totalAssets] \\
    \fundData[Holders][Funds][6][name] & \fundData[Holders][Funds][6][totalShares] & \fundData[Holders][Funds][6][change_p] & \fundData[Holders][Funds][6][totalAssets] \\
    \fundData[Holders][Funds][7][name] & \fundData[Holders][Funds][7][totalShares] & \fundData[Holders][Funds][7][change_p] & \fundData[Holders][Funds][7][totalAssets] \\
    \fundData[Holders][Funds][8][name] & \fundData[Holders][Funds][8][totalShares] & \fundData[Holders][Funds][8][change_p] & \fundData[Holders][Funds][8][totalAssets] \\
    \fundData[Holders][Funds][9][name] & \fundData[Holders][Funds][9][totalShares] & \fundData[Holders][Funds][9][change_p] & \fundData[Holders][Funds][9][totalAssets] \\
    \hline
\end{tabularx}

\subsubsection{Insider Trading}

\begin{tabularx}{\textwidth}{|X|X|X|X|X|}
    \hline
    \multicolumn{5}{|c|}{Insider Trading} \\
    \hline
    Name & Date & Amount & Price & (A)cquired or (D)isposed \\
    \hline
    \fundData[InsiderTransactions][0][ownerName] & \fundData[InsiderTransactions][0][transactionDate] & \fundData[InsiderTransactions][0][transactionAmount] & \fundData[InsiderTransactions][0][transactionPrice] & \fundData[InsiderTransactions][0][transactionAcquiredDisposed] \\
    \fundData[InsiderTransactions][1][ownerName] & \fundData[InsiderTransactions][1][transactionDate] & \fundData[InsiderTransactions][1][transactionAmount] & \fundData[InsiderTransactions][1][transactionPrice] & \fundData[InsiderTransactions][1][transactionAcquiredDisposed] \\
    \fundData[InsiderTransactions][2][ownerName] & \fundData[InsiderTransactions][2][transactionDate] & \fundData[InsiderTransactions][2][transactionAmount] & \fundData[InsiderTransactions][2][transactionPrice] & \fundData[InsiderTransactions][2][transactionAcquiredDisposed] \\
    \fundData[InsiderTransactions][3][ownerName] & \fundData[InsiderTransactions][3][transactionDate] & \fundData[InsiderTransactions][3][transactionAmount] & \fundData[InsiderTransactions][3][transactionPrice] & \fundData[InsiderTransactions][3][transactionAcquiredDisposed] \\
    \fundData[InsiderTransactions][4][ownerName] & \fundData[InsiderTransactions][4][transactionDate] & \fundData[InsiderTransactions][4][transactionAmount] & \fundData[InsiderTransactions][4][transactionPrice] & \fundData[InsiderTransactions][4][transactionAcquiredDisposed] \\
    \fundData[InsiderTransactions][5][ownerName] & \fundData[InsiderTransactions][5][transactionDate] & \fundData[InsiderTransactions][5][transactionAmount] & \fundData[InsiderTransactions][5][transactionPrice] & \fundData[InsiderTransactions][5][transactionAcquiredDisposed] \\
    \fundData[InsiderTransactions][6][ownerName] & \fundData[InsiderTransactions][6][transactionDate] & \fundData[InsiderTransactions][6][transactionAmount] & \fundData[InsiderTransactions][6][transactionPrice] & \fundData[InsiderTransactions][6][transactionAcquiredDisposed] \\
    \fundData[InsiderTransactions][7][ownerName] & \fundData[InsiderTransactions][7][transactionDate] & \fundData[InsiderTransactions][7][transactionAmount] & \fundData[InsiderTransactions][7][transactionPrice] & \fundData[InsiderTransactions][7][transactionAcquiredDisposed] \\
    \fundData[InsiderTransactions][8][ownerName] & \fundData[InsiderTransactions][8][transactionDate] & \fundData[InsiderTransactions][8][transactionAmount] & \fundData[InsiderTransactions][8][transactionPrice] & \fundData[InsiderTransactions][8][transactionAcquiredDisposed] \\
    \fundData[InsiderTransactions][9][ownerName] & \fundData[InsiderTransactions][9][transactionDate] & \fundData[InsiderTransactions][9][transactionAmount] & \fundData[InsiderTransactions][9][transactionPrice] & \fundData[InsiderTransactions][9][transactionAcquiredDisposed] \\
    \hline
\end{tabularx}

\subsection{Circle of Competence}

\subsection{Meaning}

\subsection{Moat}

\subsection{Management}

\begin{tabularx}{\textwidth}{|X|X|}
    \hline
    \multicolumn{2}{|c|}{Officers} \\
    \hline
    Title & Name \\
    \hline
    \fundData[General][Officers][0][Title] & \fundData[General][Officers][0][Name] \\
    \fundData[General][Officers][1][Title] & \fundData[General][Officers][1][Name] \\
    \fundData[General][Officers][2][Title] & \fundData[General][Officers][2][Name] \\
    \fundData[General][Officers][3][Title] & \fundData[General][Officers][3][Name] \\
    \fundData[General][Officers][4][Title] & \fundData[General][Officers][4][Name] \\
    \fundData[General][Officers][5][Title] & \fundData[General][Officers][5][Name] \\
    \fundData[General][Officers][6][Title] & \fundData[General][Officers][6][Name] \\
    \fundData[General][Officers][7][Title] & \fundData[General][Officers][7][Name] \\
    \fundData[General][Officers][8][Title] & \fundData[General][Officers][8][Name] \\
    \fundData[General][Officers][9][Title] & \fundData[General][Officers][9][Name] \\
    \hline
\end{tabularx}

\subsection{Future}

\subsection{Invert the story}

\subsection{Analyst Ratings}

\begin{tabularx}{\textwidth}{|X|X|}
 \hline
 \multicolumn{2}{|c|}{Analyst Ratings} \\
 \hline
    Rating       & \fundData[AnalystRatings][Rating]      \\
    Target Price & \fundData[AnalystRatings][TargetPrice] \\
    Strong Buy   & \fundData[AnalystRatings][StrongBuy]   \\
    Buy          & \fundData[AnalystRatings][Buy]         \\
    Hold         & \fundData[AnalystRatings][Hold]        \\
    Sell         & \fundData[AnalystRatings][Sell]        \\
    Strong Sell  & \fundData[AnalystRatings][StrongSell]  \\
 \hline
\end{tabularx}

\subsection{First Decision}
