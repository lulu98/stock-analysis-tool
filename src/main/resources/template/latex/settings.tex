% !TeX root = ./main.tex
\PassOptionsToPackage{table,svgnames,dvipsnames}{xcolor}

\usepackage[utf8]{inputenc}
\usepackage[T1]{fontenc}
\usepackage[sc]{mathpazo}
%\usepackage[ngerman,american]{babel}
\usepackage[american]{babel}
\usepackage[autostyle]{csquotes}
\usepackage[%
  backend=biber,
  url=false,
  style=alphabetic,
  maxnames=4,
  minnames=3,
  maxbibnames=99,
  giveninits,
  uniquename=init]{biblatex} % TODO: adapt citation style
\usepackage{graphicx}
\usepackage{scrhack} % necessary for listings package
\usepackage{listings}
\usepackage{lstautogobble}
\usepackage{tikz}
\usepackage{pgfplots}
\usepackage{pgfplotstable}
\usepackage{booktabs}
\usepackage[final]{microtype}
\usepackage{caption}
\usepackage[hidelinks]{hyperref}        % hidelinks removes colored boxes around references and links
\usepackage{xargs}                      % Use more than one optional parameter in a new commands
\usepackage[sort=standard]{glossaries-extra}

\usepackage{tikz}
\usepackage{adjustbox}
\usepackage{tabularx}
\usetikzlibrary{shapes.geometric,arrows,shapes,positioning}
\tikzstyle{startstop} = [rectangle, rounded corners, minimum width=3cm, minimum height=1cm,text centered, draw=black, fill=red!30]
\tikzstyle{io} = [trapezium, trapezium left angle=70, trapezium right angle=110, minimum width=3cm, minimum height=1cm, text centered, draw=black, fill=blue!30]
\tikzstyle{process} = [rectangle, minimum width=3cm, minimum height=1cm, text centered, draw=black, fill=orange!30]
\tikzstyle{decision} = [diamond, minimum width=3cm, minimum height=1cm, text centered, draw=black, fill=green!30]
\tikzstyle{arrow} = [thick,->,>=stealth]
%                            shape     border,style         distances          styles
\tikzstyle{device} =        [rectangle,draw,fill=TUMAccentBlue,   node distance=3.3cm, minimum height=1cm, text width=2cm,    text centered, rounded corners]
\tikzstyle{connectdevice} = [rectangle,draw,fill=TUMAccentOrange, node distance=3.3cm, minimum height=1cm, minimum width=2cm, text centered]
\tikzstyle{cloud} =         [ellipse,  draw,fill=red!20,    node distance=1cm, minimum height=1cm]

\bibliography{bibliography}

\setkomafont{disposition}{\normalfont\bfseries} % use serif font for headings
\linespread{1.05} % adjust line spread for mathpazo font

% Add table of contents to PDF bookmarks
\BeforeTOCHead[toc]{{\cleardoublepage\pdfbookmark[0]{\contentsname}{toc}}}

% Define TUM corporate design colors
% Taken from http://portal.mytum.de/corporatedesign/index_print/vorlagen/index_farben
\definecolor{TUMBlue}{HTML}{0065BD}
\definecolor{TUMSecondaryBlue}{HTML}{005293}
\definecolor{TUMSecondaryBlue2}{HTML}{003359}
\definecolor{TUMBlack}{HTML}{000000}
\definecolor{TUMWhite}{HTML}{FFFFFF}
\definecolor{TUMDarkGray}{HTML}{333333}
\definecolor{TUMGray}{HTML}{808080}
\definecolor{TUMLightGray}{HTML}{CCCCC6}
\definecolor{TUMAccentGray}{HTML}{DAD7CB}
\definecolor{TUMAccentOrange}{HTML}{E37222}
\definecolor{TUMAccentGreen}{HTML}{A2AD00}
\definecolor{TUMAccentLightBlue}{HTML}{98C6EA}
\definecolor{TUMAccentBlue}{HTML}{64A0C8}

% Settings for pgfplots
\pgfplotsset{compat=newest}
  % For available color names, see http://www.latextemplates.com/svgnames-colors
\pgfplotsset{cycle list={TUMBlue\\TUMAccentOrange\\TUMAccentGreen\\TUMSecondaryBlue2\\TUMDarkGray\\},}

% Settings for lstlistings
\lstset{%
  basicstyle=\ttfamily,
  columns=fullflexible,
  autogobble,
  keywordstyle=\bfseries\color{TUMBlue},
  stringstyle=\color{TUMAccentGreen}
}

% todo package configuration according to https://tex.stackexchange.com/a/178806
\usepackage[colorinlistoftodos,prependcaption,textsize=tiny]{todonotes}

\newcommandx{\unsure}[2][1=]{\todo[linecolor=red,backgroundcolor=red!25,bordercolor=red,#1]{#2}}
\newcommandx{\change}[2][1=]{\todo[linecolor=blue,backgroundcolor=blue!25,bordercolor=blue,#1]{#2}}
\newcommandx{\pending}[2][1=]{\todo[linecolor=OliveGreen,backgroundcolor=OliveGreen!25,bordercolor=OliveGreen,#1]{#2}}
\newcommandx{\improvement}[2][1=]{\todo[linecolor=Plum,backgroundcolor=Plum!25,bordercolor=Plum,#1]{#2}}
\newcommandx{\sourcerequired}[2][1=]{\todo[linecolor=Yellow,backgroundcolor=Yellow!25,bordercolor=Yellow,#1]{#2}}

\usepackage{subfigure}
\usepackage{float}