\section{Phase 1: First Impression}

The first part of the analysis should give you a short checklist to figure out if a certain
company deserves your further attention. This part of the analysis is best done in
two steps: At first, just go along these points and talk your way through them. If the
company already looks not attractive anymore, move on to another company. Otherwise,
write your thoughts out on these points.

\subsection{The Company}

Give a little summary of what the company does and sells in a few sentences. This
includes the products the company sells and its brands. This summary does not have
to be too elaborate, but should give you a first impression what awaits you with this
company.

\subsection{General Information}

Here some basic information concerning the stock and the company goes.

\subsection{Stock Categories}

Peter Lynch characterizes stocks into 6 different categories:
\begin{enumerate}
    \item Slow Growers/Sluggards: Slow growers are those stocks that have a slow
        growth rate i.e.\ a low upward slope of earnings and revenue growth. The
        growth is usually between 2-5\% CAGR and lower than the industry average.
        These sluggards may once were fast growers, but couldn’t maintain their
        growth rate as they grew bigger and can be characterized by the size and
        generosity of their dividend.\\\\
        the only reason to buy these stocks are dividends. These stocks generally
        give a decent dividend (about 2-4\%) and are a good asset during the
        recession as it’s very unlikely for their stock to crash too hard.
    \item Stalwarts: These stocks have an average growth rate as that of industry
        and are usually mid to large companies. They have an earnings growth
        rate better than the Sluggards and can be typically between the 8-12 percent
        CAGR range.\\\\
        investors can get an adequate return from these stocks if they hold
        these stocks for a long time. They generally end up being two-baggers to
        four-baggers i.e.\ they can give you 2-4 times returns in long run. Peter
        Lynch advised that it is good to have a few stalwarts in your portfolio.
    \item Fast Growers: The fast growers are everyone’s first choice. These stocks
        are generally aggressive companies and they grow at an impressive rate of
        15-25\% per year. They are fast-growth stocks and grow at a comparatively
        faster rate compared to the industry average and competitors. However,
        Peter Lynch advises that one should be open-eyed when they own a fast
        grower. There is a great likelihood for the fast growers to get hammered
        if they run out of steam or if their growth is not sustainable.
    \item Cyclicals: The Cyclical are those stocks that grow at a very fast pace
        during their favorable economic cycle. They can be distinguished from the
        fast growers as the cyclical keeps on expanding and contracting and again
        repeating the same cycle. On the other hand, the fast growers keep on
        growing consistently. The cyclical companies tend to flourish when coming
        out of a recession into a vigorous economy.\\\\
        Generally, Automobiles, Metals, Tourism, etc are examples of the cyclical
        industry. The stock charts of these companies tend to be cyclical and go
        up \& down over different phases of time.\\\\
        Peter Lynch advises the investors to own the cyclical only on the right
        part of the cycle i.e.\ when they are expanding. If bought at the wrong
        phase, it may even take them years before they perform. Timing is
        everything while investing in cyclical stocks and investors need to be
        able to detect the early signs that the industry is picking up or falling
        down.
    \item Turnarounds: The turnarounds are characterized as potential fatalities
        that have been badly hammered by the market for one or more of a variety
        of reasons but can make up the lost ground under the correct circumstances.\\\\
        Holding turnarounds can be very profitable if the management is able to
        turn the company as these stocks can be bought at a very low valuation
        by the investors. However, if the management fails to bring back the
        company on track, it can be very troublesome for the investors.
    \item Asset Plays: The asset plays are those stocks whose assets are
        overlooked by the market and are undervalued. These assets may be
        properties, equipment, or other real assets that the company is holding
        but which is not valued by the investors when there has been a general
        market downturn. The real value may be worth more than the market
        capitalization of the company.\\\\
        Many of the Public sector units are key asset plays because of the real
        estate property they are holding.\\\\
        Peter Lynch understands the worth of the asset plays. He suggests owning
        a few of these stocks in your portfolio as they are most likely to add a
        lot of value to your portfolio. However, the biggest significant factor
        while picking these stocks is to carefully estimate the right worth of
        the assets. If you are able to do it, you can pick valuable gems.
\end{enumerate}

\subsection{Shareholder Structure}

It is always interesting to see who is invested in a stock. The shareholder
structure consists of institutions, funds and insider trading.

\subsection{Circle of Competence}

You now know what the company does. Before proceeding and wasting any more
time, be honest with yourself: Do you think that you are capable of understanding the
company and the industry it operates in? If the answer is not a clear yes, look for another
industry/company. And don’t invest in something just because someone/everyone else
says so or it is a hot topic in industry, e.g. bitcoin or AI. If this company or industry is
not in your circle of competence, move on to another company.

\subsection{Meaning}

The most important thing of all is that you actually want to own the company. If you
are not absolutely sure if this industry/company aligns with your values, you should
not buy the stock and should not even bother with analyzing it any further. Don’t be
too strict but if you feel from the very start that this stock is not something for you,
don’t push it. There are thousands of other stocks that might be better suited for you.
So answer the following questions:

\begin{itemize}
    \item Does the company fit my values/interests?
    \item Why would I like to own this company?
\end{itemize}

\subsection{Moat}

Any company that you want to invest in must have a moat. This can be one of the
following moats:

\begin{itemize}
    \item Brand
    \item Secret
    \item Toll
    \item Switching
    \item Pricing
\end{itemize}

Identify the kind of moat the company is in. Make sure that the company fulfills
at least one of the moats. If the company ticks more than one box, it is even
better.

\subsection{Management}

Although good and bad management can be identified via the numbers, it is always
good to get a first impression on the people who run the company. This analysis can be
as easy as googling the CEO and listening to his voice and words. It does not have to be
a thorough and long analysis. But would you trust this person/these people to run the
company? Trust your guts! If something feels odd, move on to another company.

\subsection{Competition}

Who are the competitors of this company?

\subsection{Future}

Investing in a company only makes sense, if the company has a future. Although
this is guess work, based on the information you gained so far estimate whether the
company/industry will still be relevant in the future, say in 10 years from now. Explain
why you think so. Does the company have any goals set for the next years that it
wants to achieve? At what phase of its growth is the company right now? What is the
competitive advantage of this company compared to a competitor?

\subsection{Invert the story}

There is always something that can go wrong or something we have overlooked during
analysis. That is why it is important to understand the other side, like the competitors
etc. Ask yourself what could go wrong for the company to become a bad investment.

\subsection{Analyst Ratings}

It is also very interesting what analysts predict for the company/stock. If there
is a big difference between your prediction and the one from the analysts, then
you should figure out the reason behind the difference. Always assume that these 
specialists know something that you don't. For the rest of the time, do not let
these numbers distract from a thorough financial research.

\subsection{First Decision}

After all these previous steps, you have to make a decision if you want to proceed with
investigating this company or if you want to stop the analysis right there. If anything
does not seem right, stop right here and look for another company to analyze. But save
the progress you have made for this company. Maybe at a later point in time, you want
to deep dive into this company.\\

In case all the points above suggest that this company is a good investment, proceed
with the low-level analysis.
