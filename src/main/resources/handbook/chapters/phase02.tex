\section{Phase 2: Management}

We start the deep-dive into the company with analyzing if the company is well lead.
This process is backed up with numbers and should give us a foundation to find out if
this company is really something worth putting your money in.

\subsection{Rule 1: Low Debt}

Here, we want to find out if the debt is manageable.

\subsubsection{{Rule 1.1: Debt-to-Equity (D/E) Ratio}}

The Debt-to-Equity Ratio (D/E) should be $D/E = \frac{\text{Long Term Debt + Notes Payable}}{\text{Equity}} \leq 0.5$.
This means that the company can pay back all the money it owes (i.e. Long Term Debt
and Notes Payable) with less than half of its equity.

\subsubsection{{Rule 1.2: Debt-to-FCF (D/FCF) Ratio}}

The Debt-to-FCF ratio (D/FCF) should be $\text{D/FCF} = \frac{\text{Long Term Debt + Notes Payable}}{\text{FCF}} \leq 3$.
This means that simply by operating the business we are able to pay our debtors
in under 3 years.

\subsubsection{{Rule 1.3: Liabilities-to-Equity (L/E) Ratio}}

The Liabilities-to-Equity (L/E) should be $L/E = \frac{\text{liabilities}}{\text{equity}} \leq 0.8$.
This means that the company can pay all its liabilities with less than 80\% of its
equities.

\subsection{Rule 2: High Current Ratio}

The $\text{Current Ratio} = \frac{\text{Current Assets}}{\text{Current Liabilities}}$
should be between 1.5 and 2.5.

\subsection{Rule 3: Strong and consistent Return on Equity (ROE)}

The Return on Equity $ROE = \frac{\text{Net Income}}{\text{Equity}}$ should be
consistently above 8\% over the last years.

\subsection{Character Traits}

A great leader/CEO/management team has the following qualities:

\begin{itemize}
    \item Owner-Oriented: His personal interests directly align with the shareholders of the
          business and should be focused long-term.
    \item Honesty: If something wents wrong, he clearly describes what went wrong and what 
          can be done about it.
    \item Driven: He is driven to change the world. He uses a Big Audacious Goal (BAG) to
          motivate himself. This BAG should be easily spotted.
    \item Humility: He gives praise to others and negates his own contributions.
\end{itemize}
          
All of these characteristics should be easily identifiable.

\subsection{Red Flags}

There are also some big red flags for CEOs and management:

\begin{itemize}
    \item The CEO wants to expand his empire. Usually he unnecessarily diversifies by
buying other businesses.
    \item Huge compensation with bonus packages.
    \item Management sets unrealistic goals.
\end{itemize}

If the CEO/management has any of the described attributes, you should keep your distance.
