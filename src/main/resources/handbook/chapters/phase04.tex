\section{Phase 4: Deep-Dive (Ratio Analysis)}

This includes the ratio analysis for income statement, balance sheet and cash flow
statement. This should be the absolute last step before you would buy a stock and
should only be done if the lights for all the other key numbers are already on green.
Since this is a very laborious taks, make sure that you have the go from all the previous
steps of analysis. It should be the last check to find something in the numbers that
doesn’t add up.

\subsection{Income Statement - Ratio Analysis}

\subsubsection{Gross-Profit Margin Ratio (GPMR)}

The Gross-Profit Margin Ratio (GPMR) is calculated as
$\text{GPMR} = \frac{\text{Gross Profit}}{\text{Revenue}}$.\\
\textbf{Meaning}: If a company's GPMR is $0.70 = 70\%$, this tells us that every
time the company sells its products for 100\$, it makes 70\$ in gross profit.

\subsubsection{EBITDA Margin Ratio (EBITDA Margin)}

Earnings before interest, taxes, depreciation, and amortization (EBITDA) is calculated as
$\text{EBITDA} = \text{Earnings before interest and tax + depreciation + amortization}$.
EBITDA is an earnings measures that focuses on the essentials of a business: its
operating profitability and cash flows.\\
The EBITDA Margin Ratio is calculated as
$\text{EBITDA Margin} = \frac{\text{EBITDA}}{\text{Revenue}}$.
It is a performance metric that measures a company's profitability from
operations. That makes it easy to compare the relative profitability of two or
more companies of different sizes in the same industry. Calculating a company's
EBITDA margin is helpful when gauging the effectiveness of a company's cost-cutting
efforts. The higher a company's EBITDA margin is, the lower its operating expenses
are in relation to total revenue.\\
\textbf{Meaning}: If a company's EBITDA margin is $0.60 = 60\%$, this tells us
that every time the company sells its products for 100\$, it makes 60\$ in EBITDA.

\subsubsection{Operating Margin Ratio (OMR)}

The Operating Margin Ratio (OMR) is calculated as
$\text{OMR} = \frac{\text{Income from Operations}}{\text{Revenue}}$.\\
\textbf{Meaning}: If a company's OMR is $0.50 = 50\%$, this tells us that every
time the company sells its products for 100\$, it makes 50\$ in operating profit.

\subsubsection{Net Income Margin Ratio (NIMR)}

The Net Income Margin Ratio (NIMR) is calculated as
$\text{NIMR} = \frac{\text{Net Income}}{\text{Revenue}}$.\\
\textbf{Meaning}: If a company's NIMR is $0.30 = 30\%$, this tells us that every
time the company sells its products for 100\$, it makes 30\$ in profit.

\subsubsection{Interest Coverage Ratio (ICR)}

The Interest Coverage is defined as
$\text{ICR} = \frac{\text{Income from Operations}}{\text{Interest Expense}} \geq 5$.
This is a very important ratio for minimizing your risk. If a company cannot pay
for its interest expenses, it is heading for trouble fast.\\
\textbf{Meaning}: If a company's ICR is $30$, this tells us that the company would
be able to pay the interest expenses as much as 30 times from the operating income.

\subsection{Balance Sheet - Ratio Analysis}

\subsubsection{Profitability Ratios}

\paragraph{Return-on-Equity (ROE)}

The Return-on-Equity (ROE) is given as
$\text{ROE} = \frac{\text{Net Income}}{\text{Equity}} \geq 8\%$ for the last years.
\begin{itemize}
    \item Net Income is found on the Income Statement.
    \item Equity is found on the liability column of the Balance Sheet.
\end{itemize}
\textbf{Meaning}: If a company's ROE is $0.15 = 15\%$, this tells us that the company
has made a return of 15\$ for every 100\$ the company has retained from previous
earnings or intiial investments.

\paragraph{Return-on-Assets (ROA)}

The Return-on-Assets (ROA) is defined as
$\text{ROA} = \frac{\text{Net Income}}{\text{Total Assets}} \geq 6\%$ for the last years.
\begin{itemize}
    \item Net Income is found on the Income Statement.
    \item Total Assets is found on the asset column of the Balance Sheet.
\end{itemize}
\textbf{Meaning}: This ratio isn't that important to calculate if the company has 
very little debt, e.g.\ a debt/equity ratio of 0.5 or less. It is important for 
a company with a lot of debt. The ROA will always be lower than the ROE if the 
company has debt.

\subsubsection{Liquidity Ratios}

\paragraph{Current Ratio (CR)}

The Current Ratio (CR) is defined as
$\text{CR} = \frac{\text{Current Assets}}{\text{Current Liabilities}}$ should be
between 1.5 and 5.
\begin{itemize}
    \item Current Assets is found on the asset column of the Balance Sheet.
    \item Current Liabilities is found on the liability column of the Balance Sheet.
\end{itemize}
\textbf{Meaning}: This key ratio compares the company's expectation for cash inflow
(current assets) and cash outflow (current liabilities) during the next twelve
months. This number should be greater than 1 because if we do not get more money
in than out within the next twelve months, we will be forced to take on debt or
relinquish more equity, i.e.\ selling more stock to raise money. A current ratio
of above 5 may also indicate bad money monagament, as cash could be put ot better
use elsewhere.

\paragraph{Acid Test Ratio (ATR)}

The Acid Test Ratio (ATR) is defined as
$\text{ATR} = \frac{\text{Current Assets - Inventory}}{\text{Current Liabilities}} > 1.5$.
\begin{itemize}
    \item Current Assets is found on the asset column of the Balance Sheet.
    \item Inventory is found on the asset column of the Balance Sheet.
    \item Current Liabilities is found on the liability column of the Balance Sheet.
\end{itemize}
\textbf{Meaning}: This key ratio is also called the \textit{skeptic liquidity measure}.
It is quite conservative as we do not include the inventory. The question is: Assuming
that we do not sell anything from our inventory, do we still expect to receive
more in than we need to pay out during the next twelve months? The key ratio is
for the conservative investor. The less you know about the company, the more you
should prefer this key ratio over the current ratio.

\subsubsection{Efficiency Ratios}

\paragraph{Inventory Turnover Ratio (ITR)}

The Inventory Turnover Ratio (ITR) is defined as
$\text{ITR} = \frac{\text{Cost of Revenue}}{\text{Inventory}} > 4$.
\begin{itemize}
    \item Cost of Revenue is found on the Income Statement.
    \item Inventory is found on the asset column of the Balance Sheet.
\end{itemize}
\textbf{Meaning}: The higher the number, the more efficient the company is in
turning the inventory into sales.

\paragraph{Accounts Receivable Turnover Ratio (ARR)}

The Accounts Receivable Turnover Ratio (ARR) is defined as
$\text{ARR} = \frac{\text{Turnover}}{\text{Accounts Receivables}}$ should be
between 5 and 7.
\begin{itemize}
    \item Turnover is actually the revenues of the business found on the Income Statement.
    \item Accounts receivable is found on the asset column of the Balance Sheet.
\end{itemize}
\textbf{Meaning}: Suppose the turnover ratio is 3.41. This means if the company
makes a sale, it typicalls take $\frac{\text{365}}{3.41} = 107$ days for them to
receive payment from their customer. A higher ratio means the company gets their
money a lot faster from vendors.

\paragraph{Accounts Payable Turnover Ratio (APR)}

The Accounts Payable Ratio (APR) is defined as
$\text{APR} = \frac{\text{Cost of Revenue}}{\text{Accounts Payable}}$ should be
between 2 and 6.
\begin{itemize}
    \item Cost of Revenue is found on the Income Statement.
    \item Accounts Payable is found on the liability column of the Balance Sheet.
\end{itemize}
\textbf{Meaning}: This key ratio looks at how a company handles its credit
obligations. Suppose the turnover ratio is 2.45. This means it typicalls takes
the company $\frac{\text{365}}{2.45} = 149$ days to repay their suppliers. We want
a high accounts payable turnover ratio. A ratio of between 2 and 6 is typically
a sign of an efficient company that is satisfying bargaining power and at the same
time having no problem paying its obligations to suppliers.

\subsubsection{Solvency Ratios}

\paragraph{Debt-to-Equity Ratio (D/E)}

The Debt-to-Equity Ratio (D/E) is defined as
$\text{D/E} = \frac{\text{Long Term Debt + Notes Payable}}{\text{Equity}} < 0.5$.
\begin{itemize}
    \item Long-Term Debt is found on the liability column of the Balance Sheet.
    \item Notes Payable is found on the liability column of the Balance Sheet.
    \item Equity is found on the liability column of the Balance Sheet.
\end{itemize}
\textbf{Meaning}: There is nothing wrong with a little debt. Debt can sometimes
make things go a little faster. Too much debt, on the other hand, can undermine
the very existence of a business. Thus, this key ratio should be as as low as
possible. This key ratio comprises of the interest-bearing debt, i.e.\ the debt
we should pay interest on. This is why this key ratio is composed of long-term
debt and notes payable. This is debt that you typically have acquired from the
bank. It is the most expensive debt to obtain.\\
Suppose the D/E-ratio is $0.20 = 20\%$. This means that every time the shareholders
own 100\$ in equity, they also owe 20\$ in debt that the company is paying interest
on. This key ratio should definitely be below 0.5.

\paragraph{Liabilities-to-Equity Ratio (L/E)}

The Liabilities-to-Equity (L/E) Ratio is defined as
$\text{L/E} = \frac{\text{Total Liabilities}}{\text{Equity}} < 0.8$.
\begin{itemize}
    \item Total Liabilities is found on the liability column of the Balance Sheet.
    \item Equity is found on the liability column of the Balance Sheet.
\end{itemize}
\textbf{Meaning}: Suppose the L/E ratio is $0.47 = 47\%$. This means that every
time the shareholder has 100\$ in equity, the company would have to pay out 47\$
at some point in the future. This key ratio includes all liabilities, meaning the
intereset-bearing debt, which is the most expensive, and interest-free liabilities
such as accounts payable. It is for more conservative investors and should be
below 0.8 to be considered low-risk.

\subsection{Cash Flow Statement - Ratio Analysis}

\subsubsection{FCF-to-Revenue Ratio (FCFR)}

The FCF-to-Revenue Ratio (FCFR) is defined as
$\text{FCFR} = \frac{\text{FCF}}{\text{Revenue}} \geq 5\%$ for the last years.\\
\textbf{Meaning}: If a company has a FCFR of $0.13 = 13\%$, this tells us that
every time the company sells its products for 100\$, 13\$ will be available as
cash for the shareholders. This key ratio measures how much cash will go directly
to the owners. This means that as much as 13\$ from 100\$ of sales could be paid
directly to the shareholders as dividends.

\subsubsection{Investing-Cash-Flow-to-Operating-Cash-Flow Ratio (ICFOCF)}

The Investing-Cash-Flow-to-Operating-Cash-Flow Ratio (ICFOCF) is defined as
$\text{ICFOCF} = \frac{\text{Investing Cash Flow}}{\text{Operating Cash Flow}}$.\\
\textbf{Meaning}: If a company has a ICFOCF of $0.53 = 53\%$, this tells us every
time the company makes 100\$ in cash from its operations, 53\$ in cash are spent
on maintaining and investing in the company's growth. All this cash will only be
used for new equipment, so the investor would not get this cash out for himself.
Thus, if we have 2 companies that are equally valuable, we should prefer the one
with the lower ICFOCF key ratio. If the investing cash flow is negative, this means
that the company is investing, if the investing cash flow is positive, this means
that the company sold a formerly held asset which results in an inflow of cash.
